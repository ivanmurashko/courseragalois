%% -*- coding:utf-8 -*-
\chapter{Generalities on algebraic extensions}
We introduce the basic notions such as a field extension, algebraic
element, minimal polynomial, finite extension, and study their very
basic properties such as the multiplicativity of degree in towers. 

\section{Field extensions: examples}

\subsection{K-algebra}
\begin{definition}[K-algebra]
  Let $K$ be a field and $A$ be a \nameref{def:vectorspace} over $K$
  equipped with 
  an additional binary operation $A \times A \rightarrow A$ that we
  denote as $\cdot$ here. The the $A$ is an algebra over $K$ if the
  following identities hold $\forall x,y,z \in A$ and for every
  elements (often called as scalar) $a, b \in K$
  \begin{itemize}
  \item Right distributivity:
    $(x + y) \cdot z = x \cdot z + y \cdot z$
  \item Left distributivity:
    $z \cdot (x + y) = z \cdot x + z \cdot y$
  \item Compatibility with scalars:
    $(ax) \cdot (by) = (ab) (x \cdot y)$
  \end{itemize}
  \label{def:kalgebra}
\end{definition}

\begin{example}[Field of complex numbers $\mathbb{C}$]
  The field of complex numbers $\mathbb{C}$ can be considered as a
  K-algebra over field of real numbers $\mathbb{R}$.
  \label{ex:complexnumbers}
\end{example}

\subsection{Field extension}

Let $K$ and $L$ are fields.
\begin{definition}[Field extension]
  $L$ is an extension of $K$ if $L \supset K$
  \label{def:fextension1}
\end{definition}
and another definition
\begin{definition}[Field extension]
  $L$ is an extension of $K$ if $L$ is a K-algebra
  \label{def:fextension2}
\end{definition}
Why the 2 definitions are equivalent?

\begin{lemma}[K-algebra and \nameref{def:homomorphism}]
Given a K-algebra is the same as
having \nameref{def:homomorphism} $f: K \rightarrow A$ of rings.
\begin{proof}
Really if I have a K-algebra I can define the
\nameref{def:homomorphism} $f(k) = k \cdot 1_A$, where $1_A$ is an
identity element of $A$. Thus $k \cdot 1_A \in A$.

And conversely if I have the \nameref{def:homomorphism}
$f: K \rightarrow A$
I can define the K-algebra structure by setting
$k a = f(k) a$ because $f(k), a \in A$ and there is a multiplication
defined on $A$. As result I have a rule for multiplication a scalar
($k \in K$) on a vector ($a \in A$).
\end{proof}
\end{lemma}

\begin{lemma}[About \nameref{def:homomorphism} of fields]
Any \nameref{def:homomorphism} of fields is
\nameref{def:injection}.
\begin{proof}
Lets proof by contradiction.
Really if $f(x) = f(y)$ and $x \ne y$ then
\begin{eqnarray}
  f(x) - f(y) = 0_A,
  \nonumber \\
  f(x - y) = 0_A,
  \nonumber \\
  f(x - y) f(\left(x - y\right)^{-1}) =
  f\left(\frac{x - y}{x - y}\right) = f(1_K) = 1_A = 0_A
  \nonumber
\end{eqnarray}
that is impossible.
\end{proof}
\end{lemma}

There are some comments on the results. We have got that a
\nameref{def:homomorphism} can be set between field $K$ and its
K-algebra. This means that K-algebra is a field.
The \nameref{def:homomorphism} is \nameref{def:injection}
therefore we can allocate a sub-field $A' \subset A$ for that we will
have the \nameref{def:homomorphism} is a \nameref{def:surjection} and
therefore we have an \nameref{def:isomorphism} between original field $K$
and a sub-field $A'$. This means that we can say that the original
field $K$ is a sub-field for the K-algebra.

\begin{example}[Field extensions]
  $\mathbb{C}$ is a field extension for $\mathbb{R}$.
  $\mathbb{R}$ is a field extension for $\mathbb{Q}$
  \label{ex:fieldextension}
\end{example}

\subsection{Field characteristic}
\label{sec:fieldcharacteristic}
If $L$ is a field there are 2 possibilities
\begin{enumerate}
\item $1 + 1 + \dots \ne 0$. In this case
  $\mathbb{Z} \subset L$ but $\mathbb{Z}$ is not a field therefore $L$
  is an extension of $\mathbb{Q}$. In the case $char L = 0$
  \item $1 + 1 + \dots + 1 = \sum_{i = 1}^m 1 = 0$ for some $m \in
    \mathbb{Z}$. The first time when it happens is for a prime number
    i.e. minimal $m$ with the property is prime. In this case $char L
    = p$, where $p = min m$  - the minimal $m$ (prime) with the
    property. In this 
    case $\mathbb{Z}/p\mathbb{Z} \subset L$. The
    $\mathbb{Z}/p\mathbb{Z}$ is a field denoted by $\mathbb{F}_p$. The
    $L$ is an extension of $\mathbb{F}_p$.
\end{enumerate}
No other possibilities exist. The $\mathbb{Q}$ and $\mathbb{F}_p$ are
the prime fields. Any field is an extension of one of those.

\subsection{Field $K\left[X\right]/\left(P\right)$}
Let $K\left[X\right]$ \nameref{def:ring} of polynomials.
The $P \in K\left[X\right]$ is an irreducible. $\left(P\right)$ is an
\nameref{def:ideal} formed by the polynomial. The set of residues by
the polynomial forms a field that denoted by
$K\left[X\right]/\left(P\right)$. How we can see it?
If $Q \in K\left[X\right]$ is a polynomial that $Q \notin
\left(P\right)$ when $Q$ is prime to $P$. Then with
\nameref{lem:bezout} we can get $\exists A, B \in K\left[X\right]$
such that
\[
A P + B Q = 1,
\]
or
\[
B Q \equiv 1 \mod P,
\]
thus $B$ is $Q^{-1}$ in $K\left[X\right]/\left(P\right)$.


\section{Algebraic elements. Minimal polynomial}

\subsection{$K\left[X\right]/\left(P\right)$ field}

Alternative proof that $K\left[X\right]/\left(P\right)$ is the
\nameref{def:field}. The $\left(P\right)$ is a \nameref{def:maxideal}
but a quotient by a  \nameref{def:maxideal} is a \nameref{def:field}.

$K\left[X\right]/\left(P\right)$ is an extension of $K$ because it's
\nameref{def:kalgebra}.

\begin{example}[$\mathbb{F}_2/\left(x^2+x+1\right)$]
  Lets consider the following field
  $K = \mathbb{F}_2 = \mathbb{Z}/2\mathbb{Z} = \{0,1\}$ in the
  field polynomial $x^2+x+1$ is irreducible. It's very easy to verify
  it because $\mathbb{F}_2$ has only 2 elements that can be (possible)
  a root:
  \[
  0^2+0+1 = 1\ne 0
  \]
  and
  \[
  1^2+1+1 = 1\ne 0
  \]
  
  The polynomial has the following
  residues: $\bar{x} = x + \left(x^2+x+1\right)$ and
  $\overline{x + 1} = x + 1 + \left(x^2+x+1\right)$. Thus the field
  $\mathbb{F}_2/\left(x^2+x+1\right)$ consists of 4 elements:
  $\{0, 1, \bar{x}, \overline{x+1}\}$.

  It's easy to see that the third element ($\bar{x}$) is a root of
  $P(x) = x^2+x+1$:
  \[
  \bar{x}^2 + \bar{x} + 1 =
  P(x) + \left(P(x)\right) = \left(P(x)\right) \equiv 0 \mod P.
  \]
  
  \[
  \bar{x}^2 + \bar{x} + 1 = \bar{0},
  \]
  therefore 
  \[
  \bar{x}^2 = - \bar{x} - 1 = \bar{x} + 1 = \overline{x+1}.
  \]
  This is because we are in field $\mathbb{F}_2$ where
  \[
  2 \left(x + 1\right) \mod 2 = 0 
  \]
  and thus
  \[
  - \bar{x} - 1 = \bar{x} + 1
  \]
  
  Also
  \[
  \overline{x+1}^2 = \bar{x},
  \]
  and they are inverse each other
  \[
  \overline{x+1} \bar{x} = 1,
  \]
  \label{ex:F2overP}
\end{example}

\subsection{Algebraic elements}

\begin{definition}[Algebraic element]
  Let $K \subset L$ and $\alpha \in L$. $\alpha$ is an algebraic
  element if $\exists P \in K\left[X\right]$ such that
    $P\left(\alpha\right) = 0$. Otherwise the $\alpha$ is called
    transcendental.
  \label{def:algebraicelement}
\end{definition}

\subsection{Minimal polynomial}

\begin{lemma}[About minimal polynomial existence]
  If $\alpha$ is \nameref{def:algebraicelement} then
  $\exists!$ unitary polynomial $P$ of minimal degree such that
  $P\left(\alpha\right) = 0$. It is irreducible. $\forall Q$ such that
  $Q\left(\alpha\right) = 0$ is divisible by $P$
  \begin{definition}[Minimal polynomial]
    Such polynomial is called minimal polynomial and denoted by
    $P_{min}\left(\alpha, K\right)$.
    \label{def:minpolynomial}
  \end{definition}
  \begin{proof}
    We know that $K\left[X\right]$ is a \nameref{def:pid} and a
    polynomial $Q\left(\alpha\right) = 0$ forms an
    \nameref{def:ideal}: $I \left\{Q \in K\left[X\right] \mid
    Q\left(\alpha\right) = 0 \right\}$, so the ideal is generated by
    one element: $I = \left(P\right)$. This is an unique (up to
    constant) polynomial minimal degree in $I$.
    If $P$ is not irreducible then $\exists Q,R \in I$ such that $P = Q
    R$, $Q(\alpha) = 0$ or $R(\alpha) = 0$ and
    $deg R,Q < deg P$ that is in contradiction with the definition
    that $P$ is a polynomial of minimal degree.
  \end{proof}
\end{lemma}

\section{Algebraic elements. Algebraic extensions}

\begin{definition}
  Let $K \subset L$, $\alpha \in L$. The smallest sub-field contained
  $K$ and $\alpha$ denoted by $K\left(\alpha\right)$. The smallest
  sub-ring contained $K$ and $\alpha$ denoted by $K\left[\alpha\right]$.
\end{definition}

As soon as $K\left[\alpha\right]$ is a \nameref{def:kalgebra} it is a
\nameref{def:vectorspace} generated by $1, \alpha, \alpha^2, \dots,
\alpha^n, \dots$.

\begin{example}[$\mathbb{C}$]
  \[
  \mathbb{C} = \mathbb{R}\left(i\right) = \mathbb{R}\left[i\right]
  \]
  $\mathbb{C}$ is also a \nameref{def:vectorspace} generated by $1$
  and $i$: $\forall z \in \mathbb{Z}$ it holds $z = x + i y$ where
  $x,y \in \mathbb{R}$.
\end{example}

\begin{proposition}
  The following assignment are equivalent
  \begin{enumerate}
  \item $\alpha$ is algebraic over $K$
  \item $K\left[\alpha\right]$ is a finite dimensional
    \nameref{def:vectorspace} over $K$
  \item $K\left[\alpha\right] = K\left(\alpha\right)$ 
  \end{enumerate}
  \begin{proof}
    Lets proof that 1 implies 2. If $\alpha$ is algebraic over $K$
    then using lemma \nameref{def:minpolynomial} $\exists
    P_{min}\left(\alpha, K\right)$:
    \[
    P_{min}\left(\alpha, K\right) = \alpha^d + a_{d-1}\alpha^{d-1} +
    a_1 \alpha + a_0 = 0,
    \]
    where $a_k \in K$. Then
    \[
    \alpha^d  = - a_{d-1}\alpha^{d-1} -
    a_1 \alpha - a_0
    \]
    this means that any $\alpha^n$ can be represented as a linear
    combination of finite number of powers of $\alpha$ i.e.
    $K\left[\alpha\right]$ generated by $1, \alpha, \dots,
    \alpha^{d-1}$ is a finite dimensional \nameref{def:vectorspace}.

    Lets proof that 2 implies 3. Its enough proof that
    $K\left[\alpha\right]$ is a field. Let $x \ne 0 \in
    K\left[\alpha\right]$ then lets look at an operation
    $x \cdot K\left[\alpha\right] \rightarrow
    K\left[\alpha\right]$. This is \nameref{def:injection} because if
    $y, z \in K\left[\alpha\right]$ and $z \ne y$ then $x \cdot y \ne x
    \cdot z$. But the $K\left[\alpha\right]$ is finite dimensional
    \nameref{def:vectorspace} and a \nameref{def:homomorphism} between
    2 vector spaces with the same dimension is
    \nameref{def:surjection} thus $\exists y \in K\left[\alpha\right]$
    such that $x \cdot y = 1_{K\left[\alpha\right]}$. Therefore $x$ is
    invertable and $K\left[\alpha\right]$ is a
    \nameref{def:field}.

    Lets proof that 3 implies 1. Let $K\left[\alpha\right]$ is a
    \nameref{def:field} but $\alpha$ is not algebraic. Thus $\forall P
    \in K\left[X\right]$ $P(\alpha) \ne 0$. The we have an
    \nameref{def:injection} \nameref{def:homomorphism} $f$ :
    $K\left[X\right] \to K\left[\alpha\right]$ but $K\left[X\right]$
    is not a field thus $K\left[\alpha\right]$ should not be a field
    too that is in contradiction with the initial conditions.
  \end{proof}
  \label{prop:lec1_1}
\end{proposition}

\begin{definition}[Algebraic extension]
  $L$ an extension of $K$ is called algebraic if $\forall \alpha \in
  L$ - $\alpha$ is algebraic over $K$.
  \label{def:algebraicextension}
\end{definition}

\begin{proposition}
  If $L$ is algebraic over $K$ then any K-subalgebra of $L$ is a
  \nameref{def:field}.
  \begin{proof}
    Let $L' \subset L$ a subalgebra and let $\alpha \in L'$. We want
    to show that $\alpha$ is invertable. $\alpha$ is algebraic
    therefore $\alpha \in K\left[\alpha\right] \subset L' \subset L$
    and it's invertable.
  \end{proof}
\end{proposition}

\begin{proposition}
  Let $K \subset L \subset M$. $\alpha \in M$ - algebraic over $K$
  then $\alpha$ algebraic over $L$ and
  $P_{min}\left(\alpha, L\right)$ divides $P_{min}\left(\alpha,
  K\right)$. 
  \begin{proof}
    Its clear because $P_{min}\left(\alpha,K\right) \in
    L\left[X\right]$ thus $\exists P_L \in L\left[X\right]$ such that
    $P_L\left(\alpha\right) = 0$ i.e. $\alpha$ is algebraic over $L$.

    As soon as $P_{min}\left(\alpha,K\right) \in L\left[X\right]$ then
    $deg P_{min}\left(\alpha,L\right) \le
    P_{min}\left(\alpha,K\right)$ and as soon as
    \(
    P_{min}\left(\alpha,K\right) \in \left(P_{min}\left(\alpha,L\right)\right)
    \)
    then $P_{min}\left(\alpha,L\right)$ divides
    $P_{min}\left(\alpha,K\right)$. 
  \end{proof}
\end{proposition}


\section{Finite extensions. Algebraicity and finiteness}

\begin{definition}[Finite extension]
  $L$ is a finite extension of $K$ if $dim_k L < \infty$. $dim_k L$ is
  called as degree of $L$ over $K$ and is denoted by
  $\left[L:K\right]$
  \label{def:finiteextension}
\end{definition}

\begin{theorem}[The multiplicativity formula for degrees]
  Let $K \subset L \subset M$. Then $M$ is
  \nameref{def:finiteextension} over $K$ if and only if
  $M$ is \nameref{def:finiteextension} over $L$ and
  $L$ is \nameref{def:finiteextension} over $K$. In this case
  \[
  \left[M:K\right] = \left[M:L\right] \left[L:K\right].
  \]
  \begin{proof}
    Let $\left[M:K\right] < \infty$ but any linear independent set of
    vectors  $\left\{m_1, m_2, \dots, m_n\right\}$ over $L$ is also
    linear independent over $K$ thus
    \[
    \left[M:K\right] < \infty \Rightarrow \left[M:L\right] < \infty
    \]
    also $L$ is a vector sub space of $M$ thus if
    $\left[M:K\right] < \infty$ then $\left[L:K\right] < \infty$.

    Let $\left[M:L\right] < \infty$ and $\left[L:K\right] < \infty$
    then we have the following basises
    \begin{itemize}
    \item $L$-basis over $M$: $\left(e_1, e_2, \dots, e_n\right)$
    \item $K$-basis over $L$:
      $\left(\varepsilon_1, \varepsilon_2, \dots, \varepsilon_d\right)$
    \end{itemize}
    Lets proof that $e_i\varepsilon_j$ forms a $K$-basis over $M$.
    $\forall x \in M$:
    \[
    x = \sum_{i=1}^n a_i e_i, 
    \]
    where $a_i \in L$ and can be also written as
    \[
    a_i = \sum_{j=1}^d b_{ij} \varepsilon_j,
    \]
    where $b_{ij} \in K$.
    Thus
    \[
    x = \sum_{i=1}^n \sum_{j=1}^d b_{ij} \varepsilon_j e_i, 
    \]
    therefore $\varepsilon_j e_i = e_i \varepsilon_j$ generates $M$
    over $K$. From the other side we should check that $\varepsilon_j
    e_i$ linear independent system of vectors. Lets
    \[
    \sum_{i,j} c_{ij} \varepsilon_j e_i =
    \sum_{i=1}^n \left( \sum_{j=1}^d c_{ij} \varepsilon_j \right) e_i,
    \]
    then $\forall i$:
    \[
    \sum_{j=1}^d c_{ij} \varepsilon_j = 0.
    \]
    Thus $\forall i,j: c_{ik} = 0$ that finishes the proof the linear
    independence.   
    The number of linear independent vectors is $n \times d$ i.e.
    \[
    \left[M:K\right] = \left[M:L\right] \left[L:K\right].
    \]
  \end{proof}
  \label{thm:mulformuladegrees}
\end{theorem}

\begin{definition}[$K\left(\alpha_1, \dots, \alpha_n\right)$]
  $K\left(\alpha_1, \dots, \alpha_n\right) \subset L$ generated by
  $\alpha_1, \dots, \alpha_n$ is the smallest sub field of $L$
  contained  $K$ and $\alpha_i \in L$.
\end{definition}

\begin{theorem}[About towers]
  $L$ is finite over $K$ if and only if $L$ is generated by a finite
  number of algebraic elements over $K$.
  \begin{proof}
    If $L$ is finite then $\alpha_1, \dots, \alpha_d$ is a basis. In
    this case
    $L = K\left[\alpha_1, \dots, \alpha_d\right] = K\left(\alpha_1,
    \dots, \alpha_d\right)$. Moreover each
    $K\left[\alpha_i\right]$ is finite dimensional thus by
    proposition \ref{prop:lec1_1} $\alpha_i$ is algebraic.

    From other side if we have a finite set of algebraic elements
    $\alpha_1, \dots, \alpha_d$ then
    $K\left[\alpha_1\right]$ is a finite dimensional
    \nameref{def:vectorspace} over $K$,
    $K\left[\alpha_1, \alpha_2\right]$ is a finite dimensional
    \nameref{def:vectorspace} over $K\left[\alpha_1\right]$ and
    so on $K\left[\alpha_1, \dots, \alpha_d\right]$ is a finite dimensional
    \nameref{def:vectorspace} over
    $K\left[\alpha_1, \dots, \alpha_{d-1}\right]$. All elements
    are algebraic thus
    \[
    K\left[\alpha_1, \dots, \alpha_i\right] =
    K\left(\alpha_1, \dots, \alpha_i\right)
    \]
    Then using theorem \ref{thm:mulformuladegrees} we can conclude
    that $K\left(\alpha_1, \dots, \alpha_d\right)$ has finite
    dimension. 
  \end{proof}
  \label{thm:lec1_2}
\end{theorem}

\section{Algebraicity in towers. An example}

\begin{theorem}
  $K \subset L \subset M$ then $M$ \nameref{def:algebraicextension}
  over $K$ if and only if $M$ algebraic over $L$ and $L$ algebraic
  over $K$. 
  \begin{proof}
    If $\alpha \in M$ is an \nameref{def:algebraicelement} over $K$ then
    $\exists P \in K\left[X\right]$ such that
    $P\left(\alpha\right) = 0$ but the
    polynomial $P \in K\left[X\right] \subset L\left[X\right]$
    thus $\alpha$ is algebraic over $L$.
    If $\alpha \in L \subset M$ then $\alpha$ is algebraic over $K$
    thus $L$ is algebraic over $K$.

    Let $M$ algebraic over $L$ and $L$ algebraic over $K$ and let
    $\alpha \in M$. We want to prove that $\alpha$ is algebraic over
    $K$. Lets consider $P_{min}\left(\alpha, L\right)$ the polynomial
    coefficients are from $L$ and they (as soon as they count is a
    finite)  generate a finite extension $E$ over $K$ thus
    $E\left(\alpha\right)$ is finite over $E$ (exists a relation
    between powers of $\alpha$) and by theorem
    \ref{thm:lec1_2} is finite over $K$ thus $\alpha$ is algebraic
    over $K$.
  \end{proof}
\end{theorem}
\begin{example}[$\mathbb{Q}$ extension]
  $\mathbb{Q}\left( \sqrt[3]{2}, \sqrt{3}\right)$ algebraic and finite
  over $\mathbb{Q}$:
  \[
  \mathbb{Q} \subset \mathbb{Q}\left( \sqrt[3]{2}\right)
  \subset \mathbb{Q}\left( \sqrt[3]{2}, \sqrt{3}\right)
  \]

  Minimal polynomial
  \[
  P_{min}\left(\sqrt[3]{2}, \mathbb{Q}\right) = x^3 - 2.
  \]

  $\mathbb{Q}\left( \sqrt[3]{2}\right)$ is generated over $\mathbb{Q}$
  by $1, \sqrt[3]{2}, \sqrt[3]{4}$ thus
  $\left[\mathbb{Q}\left( \sqrt[3]{2}\right): \mathbb{Q}\right] = 3$.

  But $\sqrt{3} \notin \mathbb{Q}\left( \sqrt[3]{2}\right)$ because
  otherwise $\left[\mathbb{Q}\left( \sqrt{3}\right): \mathbb{Q}\right]
  = 2$ must devide  
  $\left[\mathbb{Q}\left( \sqrt[3]{2}\right): \mathbb{Q}\right] = 3$
  that is impossible.

  Therefore $x^2 - 3$ is irreducible over
  $\mathbb{Q}\left( \sqrt[3]{2}\right)$ and
  \[
  P_{min}\left(\sqrt{3}, \mathbb{Q}\left( \sqrt[3]{2}\right)\right) =
  x^2 - 3.
  \]

  \[
  \left[\mathbb{Q}\left( \sqrt[3]{2}, \sqrt{3}\right):
    \mathbb{Q}\right] = 3 \cdot 2 = 6.
  \]
\end{example}

\begin{proposition}[On dimension of extension]
  \[
  \left[K\left(\alpha\right) : K\right] =
  deg P_{min}\left(\alpha, K\right),
  \]
  if $\alpha$ is algebraic.
  \begin{proof}
    If $deg P_{min}\left(\alpha, K\right) = d$ then $1, \alpha,
    \cdots, \alpha^{d-1}$ - $d$ independent vectors and dimension
    $K\left(\alpha\right)$ is $d$.
  \end{proof}
  \label{prop:dimextension}
\end{proposition}

\begin{proposition}[About algebraic closure]
  If $K \subset L$ ($L$ extension of $K$). Consider
  \[
  L' = \left\{
  \alpha \in L \mid \alpha \mbox{ algebraic over } K
  \right\},
  \]
  then $L'$ sub-field of $L$ and is called as algebraic closure of $K$
  in $L$.
  \begin{proof}
    We have to prove that if $\alpha, \beta$ are algebraic then
    $\alpha + \beta$ and $\alpha \cdot \beta$ are also algebraic. This
    is trivial  because
    \[
    \alpha + \beta, \alpha \cdot \beta \in K\left[\alpha, \beta\right]
    = K\left(\alpha, \beta\right)
    \]
  \end{proof}
\end{proposition}


\section{A digression: Gauss lemma, Eisenstein criterion}

What we have seen so far:

\begin{itemize}
\item $K$ is a field, $\alpha$ is an \nameref{def:algebraicelement}
  over $K$ if it is a root of a polynomial $P \in K\left[X\right]$.
\item $L$ is an \nameref{def:algebraicextension} over $K$ if
  $\forall \alpha \in L$: $\alpha$ is an algebraic over $K$
\item $L$ is a \nameref{def:finiteextension} over $K$ if $dim_K L <
  \infty$.
\item If an extension is finite then it is algebraic
\item An extension is finite if and only if it is algebraic and
  generated by a finite number of algebraic elements (see theorem
  \ref{thm:lec1_2})
\item $\left[K\left[\alpha\right]:K\right] =
  deg P_{min}\left(\alpha, K\right)$ (see proposition
  \ref{prop:dimextension}). 
\end{itemize}

How to decide that a polynomial $P$ is irreducible over $K$?
About polynomial $x^3 - 2$ it is easy to decide that it's irreducible
over $\mathbb{Q}$, but what's about $x^{100}-2$?

\begin{lemma}[Gauss]
  Let $P \in \mathbb{Z}\left[X\right]$, i.e. a polynomial with integer
  coefficients, then if $P$ decomposes over $\mathbb{Q}$ ($P = Q\cdot
  R, deg Q,R < deg P$) then it also decomposes over $\mathbb{Z}$.
  \begin{proof}
    Let $P = Q R$ over $\mathbb{Q}$. Then
    \begin{eqnarray}
      Q = m Q_1, Q_1 \in \mathbb{Z}\left[X\right],
      \nonumber \\
      R = n R_1, R_1 \in \mathbb{Z}\left[X\right],
      \nonumber
    \end{eqnarray}
    thus
    \[
    n m P = Q_1 R_1. 
    \]
    There exists $p$ that divides $mn$: $p \mid mn$ thus in modulo $p$
    we have
    \[
    0 = \overline{Q_1}\overline{R_1}
    \]
    but $p$ is prime and the equation is in the field $\mathbb{F}_p$
    thus either $\overline{Q_1} = 0$ or $\overline{R_1} = 0$. Let
    $\overline{Q_1} = 0$ thus $p$ divides all coefficients in $Q_1$
    and we can take $\frac{Q_1}{p} = Q_2 \in
    \mathbb{Z}\left[X\right]$. Continue for all primes in $mn$ we can
    get that
    \[
    P = Q_s R_t,
    \]
    where $Q_s, R_t \in \mathbb{Z}\left[X\right]$.
  \end{proof}
  \label{lem:gauss}
\end{lemma}

\begin{example}[Eisenstein criterion]
  Lets consider the following polynomial $x^{100} -2$. It's
  irreducible. Lets prove it. If it reducible then
  $\exists Q, R \in \mathbb{Z}\left[X\right]$ such that
  \begin{equation}
    x^{100} -2 = Q R
    \label{eq:ex_eisenstein}
  \end{equation}
  Lets consider (\ref{eq:ex_eisenstein}) modulo 2. In the case we will
  have
  \[
  Q R \equiv x^{100} \mod 2,
  \]
  therefore
  \begin{eqnarray}
    Q \equiv x^k \mod 2,
    \nonumber \\
    R \equiv x^l \mod 2,
    \nonumber
  \end{eqnarray}
  or
  \[
  Q = x^k + \dots + 2 \cdot m
  \]
  and
  \[
  R = x^l + \dots + 2 \cdot n
  \]
  thus
  \[
  QR = x^{100} + 4 \cdot nm
  \]
  that is impossible because $n,m \in \mathbb{Z}$ and $nm \ne
  -\frac{1}{2}$. 
  \label{ex:eisenstein}
\end{example}

\begin{lemma}[Eisenstein criterion]
  Lets $P \in \mathbb{Z}\left[X\right]$ and
  $P = a_n X^n + a_{n-1} X^{n-1} + a_1 X + a_0$. If $\exists p$ -
  prime such that $p \nmid a_n$, $p \mid a_i \forall i < n$ and
  $p^2 \nmid a_0$, then $P \in \mathbb{Z}\left[X\right]$ is
  irreducible. 
  \begin{proof}
    the same as for example \ref{ex:eisenstein}.
  \end{proof}
  \label{lem:eisenstein}
\end{lemma}

Note: that both: \nameref{lem:gauss} and \nameref{lem:eisenstein} are
valid by replacing $\mathbb{Z}$ with an \nameref{def:ufd} $R$ and
$\mathbb{Q}$ by its factorization field.
