%% -*- coding:utf-8 -*-
\chapter{Structure of finite K-algebras continued}

We apply the discussion from the last lecture to the case of field
extensions. We show that the separable extensions remain reduced after
a base change: the inseparability is responsible for eventual
nilpotents. As our next subject, we introduce normal and Galois
extensions and prove Artin's theorem on invariants.

\section{Structure of finite K-algebras, examples (cont'd)}

Last time we have seen that a finite $K$-algebra $A$
($\left[A:K\right] < \infty$) has only finitely many maximal ideals
$m_1, \dots, m_r$ and the following equation holds (see theorem
\ref{thm:structurefinitekalgebra}):
\[
A \cong A/m_1^{k_1} \times \dots \times A/m_r^{k_r}
\]
This is a general for of \nameref{thm:chineseremainder}.

\begin{example}
  Let
  \[
  A = K\left[X\right]/\left(F\right)
  \]
  And the polynomial $F = P_1^{k_1} \dots P_r^{k_r}$ is not necessary
  reducible then by the  \nameref{thm:chineseremainder}
  \footnote{
    and theorem \ref{thm:structurefinitekalgebra}
  }
  one can get
  \[
  A \cong
  K\left[X\right]/\left(P_1 \mod F \right)^{k_1} \times \dots
  \times K\left[X\right]/\left(P_r \mod F \right)^{k_r}
  \]
\end{example}

\begin{definition}[Nilpotent element]
  Let $A$ is a \nameref{def:ring} than $x \in A$ is nilpotent if $x \ne 0$ but
  $\exists k: x^k = 0$.
  \footnote{
    Alternative definition from \cite{wiki:nilpotent}: An element,
    $x$, of a ring, $R$, is called nilpotent if there exists 
    some positive integer, $n$, such that $x^n = 0$.
  }
  \label{def:nilpotent}
\end{definition}

\begin{definition}[reduced]
  $K$-algebra $A$ is reduced if it has no \nameref{def:nilpotent}s.
  Or in other words
  \footnote{
    ??? require proof
  }
  if in the decomposition
  \[
  A \cong A/m_1^{k_1} \times \dots \times A/m_r^{k_r}
  \]
  $\forall i: k_i = 1$.
  Or 
  \footnote{
    $A/m_i$ is a field as soon as $m_i$ is a \nameref{def:maxideal}
  }
  if $A$ is a product of fields. 
  \label{def:reduced}
\end{definition}

\begin{definition}[local]
  \nameref{def:ring} $A$ is called local if it has only one
  \nameref{def:maxideal} i.e. $A \cong A/m^k$. 
  \label{def:local}
\end{definition}

If $A$ is local then all elements of $A$ are nilpotents
\footnote{
  ???
} i.e. any element of $A$ is a identity,zero or nilpotent.

Most of our last examples were examples of reduced $K$-algebras such
as
\[
\mathbb{C} \otimes_{\mathbb{R}} \mathbb{C} =
\mathbb{C} \times \mathbb{C}
\]
or
\[
\mathbb{Q}\left(\sqrt{2}\right) \otimes_{\mathbb{Q}}
\mathbb{Q}\left(i\right) =
\mathbb{Q}\left(i, \sqrt{2}\right)
\]
that is a field and if we start producing similar examples then mostly
they are reduced. Well, why? Because in fact the presence of
nilpotents has to do with inseparability. The presence of nilpotents
reflects inseparability.

So let me give you one more example. Finally of tensor product of
extensions which is not reduced. Let $K$ be a field of characteristic
$p$, for instance $\mathbb{F}_p$. Consider a field of rational
functions over $K$: $K\left(X\right)$. We will consider
$K\left(X\right)$ as an extension of $K\left(X^p\right)$ (or with new
variable $Y = X^p$ - $K\left(Y\right)$. We will be interested in
\(
K\left(X\right)
\otimes_{K\left(Y\right)}
K\left(X\right)
\) where $X$ is a $p$th root of $Y$ so
\footnote{
  as soon as
  $K\left(X\right) = K\left[T\right]/\left(T^p - Y\right)$
}
\begin{eqnarray}
K\left(X\right)
\otimes_{K\left(Y\right)}
K\left(X\right)
\cong
\nonumber \\
\cong
K\left(X\right)
\otimes_{K\left(Y\right)}
K\left[T\right]/\left(T^p - Y\right)
\cong
\nonumber \\
\cong
K\left(X\right)\left[T\right]/\left(T^p - Y\right) =
\nonumber \\
=
K\left(X\right)\left[T\right]/\left(T^p - X^p\right) =
K\left(X\right)\left[T\right]/\left(T - X\right)^p
\nonumber
\end{eqnarray}
where $T$ is another variable. As result we have got a ring with
nilpotents for example $T - X$ and of course the reason is that our
extension $K\left(X\right)$ is pure inseparable extension (see
definition \ref{def:inseparabledegree}) of $K\left(Y\right)$.  

\section{Separability and base change}
What is the reason for such a mysterious connection between 
presence of nilpotents and separability? If $L$ is separable over $K$
then the number of \nameref{def:homomorphism}s
$\left|Hom_K\left(L, \bar{K}\right)\right|$ is maximal and equal to
degree $\left[L : K\right]$ but in general it is less or equal to the
degree.  This is of course clear, because  if we have a polynomial
with distinct roots, then it's stem field for instance  has exactly
this number of  homomorphisms into the-algebraic closure and this
number is equal to the number of roots. So if some roots coincide,  
then the number of homomorphisms diminishes. 

Lets also recall \nameref{thm:basechange}. If $L$ and $E$ are
extensions of $K$ and $L$ is finite over $K$ then
\[
Hom_K\left(L, E\right) \cong
Hom_E\left(
L \otimes_K E, E
\right).
\]
In the formula, $L \otimes_K E$ is a finite $E$-algebra denoted as $A$
below.

\begin{definition}
  With \nameref{thm:chineseremainder} theorem we have
  \[
  A \cong A/m_1^{k_1} \times \dots \times A/m_r^{k_r}
  \]
  Reduced algebra $A_{red}$ is defined by the following equation
  \[
  A_{red} = A/m_1 \times \dots \times A/m_r
  \]
  \label{def:reducedalgebra}
\end{definition}

We have that
\footnote{
  ???
}
\[
A_{red} = A /\eta\left(A\right)
\]
where $\eta\left(A\right)$ is an \nameref{def:ideal} of
nilpotents in $A$.

It is clear that
\[
Hom_E\left(A, E\right) =
Hom_E\left(A_{red}, E\right)
\]
because all homomorphism into a field must be zero on all nilpotents. 

So again, we see that if there are nilpotents in the tensor product,
then there is somehow fewer space for homomorphisms. Because if A is
not reduced, then the dimension
\[
\left[A_{red} : E\right] < \left[A : E\right]
\].  
So the maximal number of homomorphisms, so let's say the slogan
``Maximal number of homomorphisms'' is attained when $A$ is reduced
and all quotients $A/m_i \cong E$.
Because those quotients are of course extensions of $E$. In general,   
those quotients are extensions of $E$. We also have
\[
A \cong A/m_1 \times \dots \times A/m_r
\]
but $Hom\left(A/m_i, E\right) = \{0\}$ if $\left[A/m_i, E\right] > 1$.  
This is because a field homomorphism is always injective. A field
homomorphism a homomorphism of fields which are extensions of $E$  an
$E$-homomorphism is injective. So you cannot map an $E$-vector space of
dimension greater than 1 into an $E$-vector space of dimension 1.

Lets take $E = \bar{K}$ then automatically we will get
$A/m_i \cong E$ because an algebraically closed field does not have a
non trivial finite extension.

So what have we had?
\[
A = L \otimes_K \bar{K},
\]
\[
A_{red} = \prod_{i=1}^r \bar{K}.
\]
The following one $A = A_{red}$ is the same to $r$ is maximal and
equal to $\left[L:K\right] = \left[A: \bar{K}\right]$. In the case
\[
r = \left|Hom_{\bar{K}}\left(A, \bar{K}\right)\right|
\left|Hom_K\left(L, \bar{K}\right)\right|
\]
So this explains why seperability is the same thing as the absence of
nilpotents. So let me formulate it as a theorem.

\begin{theorem}
  Let $L$ is a finite extension over $K$ then
  \begin{enumerate}
  \item $L$ is separable if and only if $L \otimes_K \bar{K}$
    is \nameref{def:reduced}.
    $L$ is pure inseparable if and only if $L \otimes_K \bar{K}$
    is \nameref{def:local}
  \item $L$ is separable if and only if for all algebraic extension
    $\Omega$, $L \otimes_K \Omega$ is reduced.
    $L$ is pure inseparable if and only if for all algebraic extension
    $\Omega$, $L \otimes_K \Omega$ is local.
  \item If $L$ is separable then the map
    \[
    \phi: L \otimes_K \bar{K} \to \bar{K}^n
    \]
    which sends
    \[
    \phi\left(l \otimes k\right) =
    \left(
    k \phi_1\left(l\right),
    \dots,
    k \phi_n\left(l\right)
    \right)
    \]
    where $\phi_i$ are distinct homomorphisms from $L$ to $\bar{K}$, 
    is an isomorphism.
  \end{enumerate}
  \begin{proof}
    \begin{enumerate}
    \item $L$ separable is the same thing that the algebra
      $A = L\otimes_K \bar{K}$ has $\left[L:K\right]$ factors
      $\bar{K}$ which is the same as $A$ is \nameref{def:reduced}
      since $\dim_{\bar{K}} A = \left[L:K\right]$.

      $L$ is pure inseparable: this means that exists only one
      homomorphism of $L$ into $\bar{K}$ i.e. $A$ has only one
      $\bar{K}$-homomorphism into $\bar{K}$ thus only one factor and
      as result $A$ is \nameref{def:local}.
    \item If $\Omega$ is an algebraic extension then
      \footnote {
        see definition \ref{def:embedding}
      }
      \[
      L \otimes_K \Omega \hookrightarrow L \otimes_K \bar{\Omega} =
      L \otimes_K \bar{K}.
      \]
      There is a sub-ring and so one easily checks, that a sub-ring of
      a reduced algebra is reduced and same for local.  
    \item Leave as an excises
      \footnote{
        ??? provide proof
      }
    \end{enumerate}
  \end{proof}
  \label{thm:lec5_1}
\end{theorem}
\begin{remark}
In general for modules $M$, $N$ and $P$ over a ring $R$ \textbf{not true} that
if $M \hookrightarrow N$ ($M$ is a sub module of $N$) then
$M \otimes_R P \hookrightarrow N \otimes_R P$. But this become the
truth if $R$ is a field and as result $M,N,P$ are
\nameref{def:vectorspace}s. So, for my field extensions, I can say
that if I have an extension and then I take a base change, then it
remains an extension, but  you should not think that the same thing is
true for arbitrary modules over a ring.  
\end{remark}

\section{Primitive element theorem}

\begin{definition}[Idempotent]
  The element $x$ is called idempotent if $x \cdot x = x$
  \label{def:idempotent}
\end{definition}

\begin{theorem}[Primitive element]
  Let $L$ is a finite \nameref{def:separableextension} of $K$ then it
  has only finitely many sub extensions i.e. $E$ such that $K \subset
  E \subset L$.
  \begin{proof}
    So, let's base change to $\bar{K}$
    \footnote {
      see proof of theorem \ref{thm:lec5_1} (second part of it).
    }  
    :
    \[
    E \otimes_{K} \bar{K} \hookrightarrow
    L \otimes_{K} \bar{K}
    \].
    We also have
    \[
    E \otimes_{K} \bar{K} \cong \bar{K}^m
    \]
    and
    \[
    L \otimes_{K} \bar{K} \cong \bar{K}^n
    \]
    are reduced $\bar{K}$ sub-algebras generated by
    \nameref{def:idempotent}s namely by
    $\left(0,0, \dots, 1, dots, 0\right)$ where $1$ is in $i$-th
    place.

    On the other hand $L \otimes_K \bar{K} \cong \bar{K}^n$ has only
    finitely many \nameref{def:idempotent}s because 
    $\left(a_1, \dots, a_i, \dots, a_n\right)$ is an idempotent if and
    only if all $a_i$ are 0 or 1 and therefore there
    are only finitely many ways to choose m idempotents out of them,
    so there is only finitely many ways to generate a subalgebra.  
  \end{proof}
  \label{thm:primitiveelement}
\end{theorem}

\begin{corollary}[Primitive element theorem]
  $\exists \alpha \in L$ such that $L = K\left( \alpha \right)$
  whenever $L$ is finite and separable.
  \begin{proof}
    And this is easy to see, of course, because if 
    if $L$ and $K$ are infinite, then 
    $L$ cannot be a union, a finite union of proper subextension. 
    A vector space over an infinite field is not a finite union of
    proper subspaces.   For instance a plane is not a finite union of
    lines.
    
    If $L$ and $K$ are finite, then we have  already described this
    situation completely. We have described all finite extensions  
    and have seen that they are generated by one element.
    \footnote{???}
  \end{proof}
  \label{col:primitiveelement}
\end{corollary}

\section{Examples. Normal extensions}

\subsection{Examples}

\begin{example}[Primitive element]
  \[
  \mathbb{Q}\left(\sqrt{2}, \sqrt{3}\right) =
  \mathbb{Q}\left(\sqrt{2} + \sqrt{3}\right).
  \]
  We have
  $\left[\mathbb{Q}\left(\sqrt{2} + \sqrt{3}\right) :
    \mathbb{Q}\right] = 4$ so all subextensions are quadratic. As no
  quadratic polynomial has $\alpha = \sqrt{2} + \sqrt{3}$ for a root,
  $\alpha$ generates $\mathbb{Q}\left(\sqrt{2}, \sqrt{3}\right)$.

  This must be a primitive element, generates our field. 
  It is not contained in any proper subextension. 
\end{example}

\begin{example}[Extension which cannot be generated by a single element]
  So, take $K$ equal to $\mathbb{F}_p$ and consider
  $K\left(x,y\right)$ as an
  extension of $K\left(x^p,y^p\right)$. It has degree $p^2$ i.e.
  \[
  \left[
    K\left(x,y\right) : K\left(x^p,y^p\right)
    \right] = p^2.
  \]
  We have $\forall \alpha \in K\left(x,y\right) \setminus
  K\left(x^p,y^p\right)$ is of degree $p$ over
  $K\left(x^p,y^p\right)$. This is because
  $\alpha^p \in K\left(x^p,y^p\right)$. So, no element like these can
  generate our extension.  
\end{example}

\subsection{Normal extensions}

\begin{definition}[Normal extension]
  A normal extension of $K$ is a \nameref{def:splittingfield} of a
  family of polynomials in $K\left[X\right]$.
  \label{def:normalextension}
\end{definition}

\begin{remark}[Normal extension]
  So, take a bunch of polynomials in $K$ and we adjoin all their roots
  to $K$, and this is what is called a normal extension.
  For instance, a \nameref{def:splittingfield} of one polynomial is
  also a normal extension. 
\end{remark}

\begin{theorem}
  The following conditions are equivalent for an extension $L$ of $K$:
  \begin{enumerate}
  \item $\forall x \in L$ $P_{min}\left(x, K\right)$ splits in $L$.
  \item $L$ is \nameref{def:normalextension}
  \item All \nameref{def:homomorphism}s from $L$ to $\bar{K}$ have the
    same image.
  \item The \nameref{def:group} of \nameref{def:automorphism}s
    $Aut\left(L/K\right)$ acts transitively on this set of
    homomorphisms $Hom_K\left(L, \bar{K}\right)$.
  \end{enumerate}
  \begin{proof}
    1 implies 2: Take
    $\left(P_i\right)_{i \in I} = \left\{P_{min}\left(x, K\right) \mid
    x \in L\right\}$ - the set of polynomials.
    $L$ will be a splitting field of the set $\left(P_i\right)_{i \in I}$.

    2 implies 3: Let
    $S = \left\{\mbox{roots of } P_i, i \in I \mbox{ in } L\right\}$
    and
    $S' = \left\{\mbox{roots of } P_i, i \in I \mbox{ in }
    \bar{K}\right\}$ then any homomorphism
    $\phi: L \to \bar{K}$ sends $S$ to $S'$, but $S$ generates $L$
    over $K$, so $\phi\left(S\right)$ determines $\phi\left(L\right)$.

    3 implies 4: Let $j, j' \in Hom_K\left(L, \bar{K}\right)$ then
    they send $L$ isomorphically to its image. So, these are
    isomorphisms from $L$ to $L'$. So
    \[
    L \xrightarrow[j']{} L' \xrightarrow[j^{-1}]{} L,
    \]
    take $j^{-1} \cdot j' \in Aut \left(L/K\right)$ and it sends $j$
    to $j'$.

    4 implies 1:  I have this transitive action and I have to prove
    that any minimal polynomial splits. Consider
    $P_{min}\left(x, K\right)$. $\alpha_1, \dots, \alpha_n$ - roots in
    $\bar{K}$. Then I have map $K\left(x\right) \to
    K\left(\alpha_i\right)$ that extends to
    $j_i: L \xrightarrow[x \to \alpha_i]{} \bar{K}$. This is by
    theorem \nameref{thm:lec2_3}.
    $\exists \theta_i \in Aut\left(L/K\right)$ such that
    $j_1 \theta_i = j_i$ thus
    $\alpha_i \in j_1\left(L\right)$ or all roots are in
    $j_1\left(L\right)$ and the polynomial
    $P_{min}\left(x, \bar{K}\right)$ splits over $j_1\left(L\right)$
    but this means that it splits over $L$    
  \end{proof}
  \label{thm:lec5_3}
\end{theorem}

\section{Galois extensions}

Now we are ready to give a definition for central object of Galois
theory

\begin{definition}[Galois extension]
  A Galois extension is a \nameref{def:normalextension} and
  \nameref{def:separableextension}.
  \label{def:galoisextension}
\end{definition}

\section{Artin's theorem}
