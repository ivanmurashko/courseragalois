%% -*- coding:utf-8 -*-
\chapter{Ring extensions, norms and traces, reduction bp}
We build a tool for finding elements in Galois groups, learning to use
the reduction modulo $p$. For this, we have to talk a little bit about
integral ring extensions and also about norms and traces.


\section{Integral elements over a ring}

Let $P \in \mathbb{Z}\left[X\right]$. We want to know what is
$Gal\left(P\right)$. Just a reminder that $Gal\left(P\right) =
Gal\left(K/\mathbb{Q}\right)$ where $K$ is a
\nameref{def:splittingfield} of $P$. We have already done the work for
several types of polynomials: cyclotomic polynomial,
\nameref{sec:kummerextension} and so on.

Sometimes if, our polynomial is a kind of combination of then the
explicit information about the roots helps to calculate the Galois
group. For instance if we have polynomial $X^5 - 2$ we know it's
roots: $\sqrt[5]{2}, j^k \sqrt[5]{2}$, where $j= e^{\frac{2 \pi
    i}{5}}, 1 \le k \le 4$. Now we have a lot about
\nameref{def:galoisgroup}. If $K$ is the splitting field of the
polynomial then we have the following towers:

  \begin{tikzpicture}[descr/.style={fill=white,inner sep=2.5pt}]
    \matrix (m) [matrix of math nodes, row sep=3em,
      column sep=3em]
            { & \mathbb{Q}\left(j\right) & \\
              \mathbb{Q} & & \mathbb{Q}\left(\sqrt[5]{2}, j\right) = K\\
              & \mathbb{Q}\left(\sqrt[5]{2}\right) & \\ };
            \path[->,font=\scriptsize]
            (m-2-1) edge node[descr] {$ 4 $} (m-1-2)
            (m-1-2) edge node[descr] {$ 5 $} (m-2-3)
            (m-2-1) edge node[descr] {$ 5 $} (m-3-2)
            (m-3-2) edge node[descr] {$ 4 $} (m-2-3);
  \end{tikzpicture}
  
From that we know we can conclude that it follows that our Galois
group, contains a normal cyclic subgroup of a order of five
$\mathbb{Z}/5\mathbb{Z}$. And then 
the quotient is the Galois group of cyclotomic extension, so this is
$\left(\mathbb{Z}/5\mathbb{Z}\right)^\times$. So this is a group of
order 20. You can show that this 
is noncommutative, and from this exact sequence, you have some
information about it.   

\section{Integral extensions, integral closure, ring of integers of a number field}
\section{Norm and trace}
\section{Norm and trace (cont'd). Ring of integers is a free module}
\section{Reduction modulo a prime}
\section{Reduction modulo a prime and finding elements in Galois groups}
