%% -*- coding:utf-8 -*-
\chapter{Solvability by radicals, Abel's theorem. A few words on
  relation to representations and topology}

We finally arrive to the source of Galois theory, the question which
motivated Galois himself: which equation are solvable by radicals and
which are not? We explain Galois' result: an equation is solvable by
radicals if and only if its Galois group is solvable in the sense of
group theory. In particular we see that the "general" equation of
degree at least 5 is not solvable by radicals. We briefly discuss the
relations to representation theory and to topological coverings.

\section{Extensions solvable by radicals. Solvable groups. Example} 

\subsection{Extensions solvable by radicals}

Let $K$ is a field of characteristic 0: $char K = 0$. It is embedded
into its \nameref{def:algebraicclosure}.

\begin{definition}[Extension solvable by radicals]
  A finite extension $E$ of $K$ is solvable by radicals if
  $\exists \alpha_1, \dots, \alpha_r$ generating $E$ such that
  $\alpha_i^{n_i} \in K\left(\alpha_1, \dots, \alpha_{i-1}\right)$ for
  some $n_i \in \mathbb{N}$.
  \label{def:solvableextension}
\end{definition}

\begin{example}
  Let $K = \mathbb{Q}$, $E = \mathbb{Q}\left(\sqrt[3]{2 + 3
    \sqrt{7}}, \sqrt[5]{4 + 5 \sqrt{11}}\right)$. We have
    $\alpha_1 = \sqrt{7}, \alpha_2 = \sqrt{11},
    \alpha_3 = \sqrt[3]{2 + 3\sqrt{7}}, \alpha_4 = \sqrt[5]{4 + 5
      \sqrt{11}}$. 
\end{example}

\begin{definition}[Polynomial solvable by radicals]
  $P \in K\left[X\right]$ is called solvable by radicals if exists a
  $E$ - \nameref{def:solvableextension} and containing all roots of
  $P$. 
  \label{def:solvablepolynomial}
\end{definition}
So more precisely, it would say that the equation, $P = 0$ is solvable
by radicals.

\begin{property}
  \begin{enumerate}
  \item \nameref{def:compositeextension} of solvable by radicals is itself
    solvable by radicals
  \item If $L$ extension of $K$ is solvable by radicals (by definition
    $L$ should be finite extension of $K$) then exists a finite
    \nameref{def:galoisextension} $E$ containing $L$ and solvable by
    radicals.    
  \end{enumerate}
  \begin{proof}
    For the first property: ???
    
    For the second property: Indeed take a composite of all images of $L$ in
    $\bar{K}$. Or those are the same as images of $L$ by
    $Gal\left(\bar{K}/K\right)$ 
  \end{proof}
  \label{property:solvable}
\end{property}

\subsection{Solvable groups}
This shall be a brief reminder since this is not a course on group
theory, you are supposed to know some group theory already. So I
somehow I presume that you are familiar with this definition but I
will recall the definition of basic properties.

\begin{definition}[Solvable group]
  $G$ is called solvable if it has a filtration 
  i. e. $G = G_0 \supset G_1 \supset \dots \supset G_{r-1} \supset G_r
  = \left\{e\right\}$, such that $G_i$ is a normal subgroup of
  $G_{i-1}$ and the \nameref{def:quotientgroup} $G_{i-1}/G_i$ is abelian.
  \label{def:solvablegroup}
\end{definition}

\begin{example}[Group of permutations $S_3$]
  Consider $S_3$ - the group of permutations (see also example
  \ref{ex:s3group}). It's solvable because 
  $S_3 \supset A_3 \supset \left\{e\right\}$.

  We have $\left|S_3/A_3\right| = 2$ (see example
  \ref{ex:s3a3quotientgroup}) i.e. $S_3/A_3$ is cyclic of order 
  2. $\left|A_3\right|$ i.e. $A_3$ - cyclic of order 3.
  \label{ex:lec8_s3}
\end{example}

\begin{example}[Group of permutations $S_4$]
  Consider $S_4$ - the group of permutations (see also example
  \ref{ex:s3group}). It's solvable because 
  $S_4 \supset A_4 \supset K \supset \left\{e\right\}$, where $K$ -
  is a subgroup of double transpositions (see example
  \ref{ex:permutation} for permutation cycles notation):
  \[
  K = \left\{
  e, (12)(34), (13)(24), (14)(23)
  \right\}.
  \]
  A double transposition is a product of two transpositions with
  distinct support, right, which permute the distinct elements.

  $A_4 \triangleleft S_4$, $\left|S_4/A_4\right| = 2$, i.e.
  $S_4/A_4$ is cyclic of order 2.

  $K \triangleleft A_4$, $\left|A_4/K\right| = 3$, i.e.
  $A_4/K$ is cyclic of order 3.

  $K$ is \nameref{def:abeliangroup} and
  $K \cong \mathbb{Z}/2\mathbb{Z} \times \mathbb{Z}/2 \mathbb{Z}$.

  So this shows that $S_4$ is solvable.
  \label{ex:lec8_s4}
\end{example}

\section{Properties of solvable groups. Symmetric group}

\begin{property}
  If $G$ is solvable and $H \subset G$ is a subgroup of $G$ then $H$
  is solvable.
  \begin{proof}
    Indeed $G_i \cap H$ gives a filtration with required property.
  \end{proof}
  \label{property:lec8_solvable1}
\end{property}

\begin{property}
  If $G$ is solvable and $H \triangleleft G$ is a normal subgroup of
  $G$ then $G/H$ is solvable.
  \begin{proof}
    Indeed consider a projection map
    \begin{equation}
      \pi: G \to G/H
      \label{eq:lec8_solvable_pi}
    \end{equation}
    then $\pi\left(G_i\right)$ gives a filtration $\left(G/H\right)_i$
    on $G/H$ with required properties.
  \end{proof}
  \label{property:lec8_solvable2}
\end{property}

\begin{property}
  If $H \triangleleft G$, $H$ and $G/H$ are solvable then $G$ is
  solvable. 
  \begin{proof}
    Put togeter the filtration $H_i$ and
    $\pi^{-1}\left(\left(G/H\right)_j\right)$ (see
    (\ref{eq:lec8_solvable_pi}) for $\pi$ definition).
  \end{proof}
  \label{property:lec8_solvable3}
\end{property}

\begin{property}
  If $G$ is finite than $G$ is solvable (i.e. has a finite filtration with
  Abelian quotients) if and only if there exists a
  finite filtration with cyclic quotients.
  \begin{proof}
    This is just because a finite \nameref{def:abeliangroup} is just a
    product of cyclic groups.
  \end{proof}
  \label{property:lec8_solvable4}
\end{property}

Lets also look at another definition of solvable group
\begin{definition}[Solvable group]
  $G$ is called solvable if the following sequence is finite:
  \[
  G
  \supseteq \left[G, G\right] = G^{(1)}
  \supseteq \left[G^{(1)}, G^{(1)}\right] = G^{(2)}
  \supseteq \dots \supseteq
  \left[G^{(n-1)}, G^{(n-1)}\right] = G^{(n)} = \left\{e\right\}
  \]
  where $G^{(i)} = \left[G^{(i-1)}, G^{(i-1)}\right]$ is the
  \nameref{def:commutatorsubgroup}.
  \label{def:solvablegroupadd}
\end{definition}

\section{Galois theorem on solvability by radicals}
\section{Examples of equations not solvable by radicals."General equation"}
\section{Galois action as a representation. Normal base theorem}
\section{Normal base theorem (cont'd). Relation with coverings}
